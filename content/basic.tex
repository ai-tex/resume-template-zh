% basic.tex - 基本信息(编辑此文件更新姓名、联系方式与意向)
% 用途:存放姓名、联系方式、意向、头像等基本信息。
% 说明:主要字段放在下方的宏中。可选字段默认被注释,需时取消注释并填写。

% ====== 必填 / 主要信息(请编辑) ======
\newcommand{\ProfileName}{瓦豆鲁迪 (Kirby)}
\newcommand{\ProfilePhone}{138-0000-0000}
\newcommand{\ProfileEmail}{kirby@example.com}
% 联系方式宏(用于页眉与 resumetitle)
\newcommand{\ProfileContact}{电话:\ProfilePhone\\ 邮箱:\href{mailto:\ProfileEmail}{\ProfileEmail}}
% 求职意向(示例:虚构角色,示例用途)
\newcommand{\ProfileTarget}{虚拟角色(示例)}
% 申请研究/方向(示例)
\newcommand{\ProfileDirection}{星之任务与救援}

% ====== 可选信息(默认注释——需要时取消注释并填写) ======
% 个人网站 / 作品集(可选)
% \newcommand{\ProfileWebsite}{https://www.github.com}
% GitHub(可选)
% \newcommand{\ProfileGitHub}{https://github.com/neverbiasu}
% 现居地(可选)
% \newcommand{\ProfileLocation}{上海市}
% 籍贯(可选)
% \newcommand{\ProfileHometown}{浙江杭州}
% 出生年月(可选)
% \newcommand{\ProfileDOB}{1995.08}

% 如需显示头像,请在此处设置路径(相对工作区),例如:
% 在本文件中取消下面一行的注释并设置路径(例如:assets/avatar.jpg):
% \newcommand{\ProfilePhoto}{assets/avatar.jpg}
% 如果你的 avatar.png 文件损坏或不是有效的 PNG/JPG,会导致 xelatex 构建失败。
% 将下面一行取消注释并替换为有效图片路径可显示头像:
\newcommand{\ProfilePhoto}{assets/WaddleDee.png}

% ====== 渲染宏:在 main.tex 中调用 \BasicSection ======
% 名称大号加粗,联系方式在同一行,下面列出求职意向与申请方向。
\newcommand{\BasicSection}{%
	% Layout: left = name/info (wide), right = photo (narrow)
	\begin{minipage}[t]{0.80\textwidth}
		{\LARGE\bfseries \ProfileName\par}
		\vspace{4pt}
		{\normalsize 电话:\ProfilePhone \quad 邮箱:\href{mailto:\ProfileEmail}{\ProfileEmail}\par}
		% 可选:显示 GitHub / 网站(只有当宏被定义且未注释时显示)
		\ifdefined\ProfileGitHub
			{\normalsize GitHub:\href{\ProfileGitHub}{\ProfileGitHub}\par}
		\fi
		\ifdefined\ProfileWebsite
			{\normalsize 个人网站:\href{\ProfileWebsite}{\ProfileWebsite}\par}
		\fi
		\vspace{6pt}
		{\normalsize \textbf{求职意向:} \ProfileTarget \quad \textbf{申请方向:} \ProfileDirection\par}
		% 可选信息(现居地 / 籍贯 / 出生年月)
		\ifdefined\ProfileLocation
			\vspace{4pt}{\normalsize 现居地:\ProfileLocation\par}
		\fi
		\ifdefined\ProfileHometown
			{\normalsize 籍贯:\ProfileHometown\par}
		\fi
		\ifdefined\ProfileDOB
			{\normalsize 出生年月:\ProfileDOB\par}
		\fi
	\end{minipage}\hfill
	\begin{minipage}[t]{0.18\textwidth}
		\raggedleft
		% Photo slot: safe include on document compile
		\IfFileExists{\ProfilePhoto}{%
			\includegraphics[width=\linewidth,keepaspectratio]{\ProfilePhoto}
		}{% if not exist, leave blank
		}
	\end{minipage}
}

% 额外说明:可选信息默认注释,不会显示;若希望开启某一项,取消对应行的注释并填写内容即可。
